
\begin{table}[hpbt]
\centering

\begin{tabular}{clrr}
	Bit
	&
	\signal{signal}, \protein{txn factor}, \promoter{promoter}
	&
	Primary reference
	&
	Details

	\\
	
	\hline
	
	\ce{w_A}
	& 
	$
		\href{https://pubchem.ncbi.nlm.nih.gov/compound/L-Arabinose}{\signal{Ara}}
		\act
		\protein{AraC}
		\act
		\promoter{BAD}
	$
	& 
	\cite{Schleif2000} % move ref to details section
	\footnote{\tiny{propose to use Nielsen for Ara and IPTG b/c of reduced crosstalk, see \S\ref{ss:wAB} \hh{okay}}}
	& 
	\S\ref{ss:wAB}/p.\pageref{ss:wAB}
	
	\\
	
	\ce{w_B}
	&
	$
		 \href{https://pubchem.ncbi.nlm.nih.gov/compound/656894}{\signal{IPTG}}
		 \rep
		 \protein{LacI}
		 \rep
		 \promoter{Tac}
	$
	&
	\TODO{ref}
	&
	\S\ref{ss:wAB}/p.\pageref{ss:wAB}
	
	\\
	
	\ce{r_0}
	&
	$
		 \href{https://pubchem.ncbi.nlm.nih.gov/compound/119133}{\signal{{3OC6}{-}{HSL}}}
		 \act
		 \protein{LuxR}
		 \act
		 \promoter{Lux}
	$
	&
	\cite[\href{https://www.embopress.org/doi/full/10.15252/msb.20156590}{p.1}]{Grant2016}%
	\footnote{temporary note: \hh{this paper proposed the orthogonality between 3OC6 and 3OC12}}
	&
	\S\ref{ss:3OC6}/p.\pageref{ss:3OC6}
	
	\\
	
	\ce{r_1}
	&
	$
		\href{https://pubmed.ncbi.nlm.nih.gov/21949379/}{\signal{IV{-}HSL}}
		\rep
		\protein{BjaR}
		\rep
		\promoter{Bja}
	$
	&
	\cite[\href{https://www.nature.com/articles/s41467-020-17993-w\#Sec23}{SM}:p.2]{DuETAL2020}
	\marginnote{\tiny why did you change this ref?}
	&
	\S\ref{ss:IV}/p.\pageref{ss:IV}

	\\
	
	\ce{s_0}
	&
		$
		\href{https://en.wikipedia.org/wiki/2,4-Diacetylphloroglucinol}{\signal{DAPG}}
		\rep
		\protein{PhlF}
		\rep
		\promoter{PhlF}
	$
	&
	\TODO{ref}
	&
	\S\ref{ss:DAPG}/p.\pageref{ss:DAPG}
	
	\\
	
	\ce{c_1}
	&
	$
		\signal{Sal}
		\act
		\protein{NahR}
		\act
		\promoter{Sal}
	$
	&
	\TODO{ref}
	&
	\S\ref{ss:Sal}/p.\pageref{ss:Sal}
   
	\\
	
	\ce{s_1}
	&
	$
		\href{https://pubchem.ncbi.nlm.nih.gov/compound/N-_4-Coumaroyl_-L-homoserine-lactone}{\signal{pC{-}HSL}}
		\act
		\protein{RpaR}
		\act
		\promoter{Rpa}
	$
	&
	\TODO{ref}
	&
	\S\ref{ss:pC}/p.\pageref{ss:pC}
	
	\\
	
	\ce{c_2}
	&
	$
		\signal{MMF}
		\rep
		\protein{MmfR}
		\rep
		\promoter{Mmf}
	$
	&
	\cite[\href{https://www.nature.com/articles/s41467-020-17993-w\#Sec23}{SM}:p.2]{DuETAL2020}
	%\cite{ORourkeETAL2009}
	&
	\S\ref{ss:MMF}/p.\pageref{ss:MMF}
	
	\\
	
	\ce{s_2}
	&
	$
		\href{http://www.chemspider.com/Chemical-Structure.2497481.html}{\signal{{3OC12}{-}HSL}}
		\act
		\protein{LasR}
		\act
		\promoter{LasR}
	$
	\hh{act}
	\ra{why?}
	% [Du] isn't clear about it
	% Generally, we need constructive references, ie containing the actual sequences
	% [Du] has the sequences but they don't use "standard" constructs
	% Hence the star
	%
	% I'm /for/ reviews and the original publications 
	% but they should go to the `details` section
	&
	\cite[\href{https://www.nature.com/articles/s41467-020-17993-w\#Sec23}{SM}:p.3]{DuETAL2020}
	&
	\S\ref{ss:3OC12}/p.\pageref{ss:3OC12}
	
	\\
	
	\ce{c_3}
	&
	$
		\signal{NG}
		\act
		\protein{FdeR}
		\act
		\promoter{FdeA}
	$
	\hh{act}
	&
	\cite[\href{https://www.nature.com/articles/s41467-020-17993-w\#Sec23}{SM}:p.3]{DuETAL2020}
	&
	\S\ref{ss:NG}/p.\pageref{ss:NG}
\end{tabular}

\caption{%
	Extra-/intercellular \signal{signals}
	with downstream 
	\protein{factors}
	and
	\promoter{promoters}.
}
%
\label{t:signals}


\TODO{what is \href{https://en.wikipedia.org/wiki/Salicylic_acid}{Sal}?}

\TODO{what is MMF?}

\TODO{better link for IV-HSL}
	
\TODO{
what is NG? 
is it 
\href{https://en.wikipedia.org/wiki/Naringenin}{Naringenin}?
cf.~\cite{SiedlerETAL2014}
} \hh{Yes. You may also refer to the link \url{http://parts.igem.org/Part:BBa_K1497019}}

\TODO{check that the stars are right} \hh{AraC is correct while Las^*}
\end{table}
