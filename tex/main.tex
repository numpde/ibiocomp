\documentclass[12pt]{article}


\usepackage{amsmath, amsfonts, amssymb}
\usepackage{graphicx}
\usepackage[svgnames]{xcolor}
\usepackage{datetime2}
\usepackage{epstopdf}
\usepackage[
	colorlinks=true, 
	citecolor={DarkRed}, urlcolor={DarkBlue}, linkcolor={DarkBlue},
]{hyperref}

% http://ctan.math.washington.edu/tex-archive/macros/latex/contrib/mhchem/mhchem.pdf
\usepackage[version=3]{mhchem}

\usepackage{fullpage}
\setlength{\parindent}{0cm}

% For editing purposes:
%\usepackage[margin=10pt]{geometry}

% https://latex.org/forum/viewtopic.php?t=10456
\usepackage{titlesec}
\titleformat{\subsubsection}[runin]% runin puts it in the same paragraph
{\normalfont\bfseries}% formatting commands to apply to the whole heading
{\thesubsubsection}% the label and number
{}% space between label/number and subsection title
{}% formatting commands applied just to subsection title
[.]% punctuation or other commands following subsection title


\newcommand{\TODO}[1]{\textrm{\color{red}TODO: #1}}

% https://bitbucket.org/goodnightmath/covariance/src/master/tex/main.tex

%%%%%%%%%%%%%%%%%%%%%%%%%%%%%%%%%%%%%%%%%%%%%%%%%%%%%%%%%%%%%%%%%%%%%%%%%%%%%%%
% http://tex.stackexchange.com/a/106577/44073
\usepackage{ifthen}
\newcounter{todoindex}\setcounter{todoindex}{0}
\newcommand\ADDTODO[1]{%
	\addtocounter{todoindex}{1}%
	\expandafter\gdef\csname todo\roman{todoindex}\endcsname{#1}%
	%\expandafter\csname todolabel\roman{todoindex}\endcsname
	\label{todolabel\roman{todoindex}}
}
\renewcommand\TODO[1]{%
	{%
		\ADDTODO{#1}%
		{\textrm{\color{red}TODO(\arabic{todoindex}): #1}}%
	}%
}
\newcommand\CHECK[1]{%
	\ADDTODO{CHECK CLAIM: {#1}}%
	{\color{toverify}#1}%
	\smash{\marginnote{\text{\color{red}*}}}%
}
\newcounter{indextodo}
\newcommand{\SHOWTODOS}{%
	\setcounter{indextodo}{0}%
	\begin{enumerate}
	\item[{\color{red} TODOs:}]
	\whiledo{\value{indextodo} < \value{todoindex}}{%
		\addtocounter{indextodo}{1}%
		\item[\color{red}\arabic{indextodo}.]
		p.\pageref{todolabel\roman{indextodo}}.
		%
		\csname todo\roman{indextodo}\endcsname
	}%
	\end{enumerate}
}
%%%%%%%%%%%%%%%%%%%%%%%%%%%%%%%%%%%%%%%%%%%%%%%%%%%%%%%%%%%%%%%%%%%%%%%%%%%%%%%



\renewcommand{\d}{\mathrm{d}}

\newcommand{\TEXT}[1]{\quad\text{#1}\quad}
\newcommand{\with}{\text{ $:$ }}

\newcommand{\cbra}[1]{{\color{gray}\ensuremath{#1}}}
\newcommand{\signal}[1]{\ensuremath{\cbra{\langle}\mathrm{#1}\cbra{\rangle}}}
\newcommand{\protein}[1]{\ensuremath{\cbra{(}\mathrm{#1}\cbra{)}}}
\newcommand{\promoter}[1]{\ensuremath{\cbra{[}\mathrm{#1}\cbra{]}}}

% https://tex.stackexchange.com/questions/543953/arrow-with-blunted-end-head-in-math-mode
\newcommand{\act}{\ensuremath{\to}}
\newcommand{\rep}{\ensuremath{\mathrel{\raisebox{-.3ex}{\rotatebox{90}{\scalebox{1}[1.2]{$\bot$}}}}}}

\def\[#1\]{\begin{align}#1\end{align}}

% https://tex.stackexchange.com/questions/114113/how-to-label-text-with-equation-number
\newcommand{\eqnum}{\leavevmode\hfill\refstepcounter{equation}\textup{{(\theequation)}}}



\begin{document}

\clearpage

\section{Introduction}

\subsection{The game of 15 sticks}

Two players start with 15 sticks.
Each player takes 1, 2 or 3 sticks.
The last to take a stick loses.



\section{Genetic toolbox}


\TODO{Specify initial conditions}


\subsection{Transducers}


\subsubsection*{AmtR}

\TODO{MoA}

\cite{JakobyETAL2000}


\subsection{Sensors}

\subsubsection*{3OC6-HSL/LuxR}

Pathway:
\[
	\signal{3OC6\text{-}HSL} \act \protein{LuxR} \act \promoter{Lux} \act \protein{Output}
	.
\]

 
The transcription activator \protein{LuxR}
occurs in Gram-negative bacteria
such as \emph{Vibrio fischeri}.
%
The bacterium is permeable to the (auto)inducer,
here \signal{3OC6\text{-}HSL}.
%
The inducer binds to the N-terminal of \protein{LuxR},
which otherwise inhibits its
functional C-terminal \cite{StevensDolanGreenberg1994}.
%
%
The purified C-terminal binds 
upstream of the \emph{lux} box 
(which is centered at $-42.5$bp \cite{EglandGreenberg1999});
however, 
together with the RNA Pol,
it protects the \emph{lux} box and the \emph{lux} operon
promoter
\cite{StevensDolanGreenberg1994}.
%
%
%
We abbreviate
$\protein{LuxR^\star} := \signal{3OC6\text{-}HSL} \with \protein{LuxR}$
and
$\promoter{Lux^\star} := \protein{LuxR^\star} \with \promoter{Lux}$.
%
%\TODO{
%Let $\protein{LuxR^\star}$ denote 
%the activated form
%$\signal{3OC6}:\protein{LuxR}$.
%}
%
%
We assume the mechanism:
%
\begin{subequations}
\[
	\label{e:LuxR_Act}
	\cee{
		\signal{3OC6\text{-}HSL} + \protein{LuxR}
		& <=>>
		\protein{LuxR^\star}
	}
	\\
	\label{e:P_Lux_Act}
	\cee{
		\protein{LuxR^\star} + \promoter{Lux}
		& <=>>
		\promoter{Lux^\star}
	}
	\\
	\label{e:P_Lux_Out}
	\cee{
		\promoter{Lux^\star}
		& ->
		\promoter{Lux^\star} + \protein{Output}
	}
	.
\]
\end{subequations}
%\TODO{
%\[
%	\signal{3OC6} + \protein{LuxR}
%	& \longleftrightarrow
%	\protein{LuxR^\star}
%	\\
%	\protein{LuxR^\star} + \protein{RNApol} + \promoter{Lux}
%	& \longleftrightarrow
%	\protein{LuxR^\star} + \protein{RNApol} : \promoter{Lux}
%	\\
%	\protein{RNApol} : \promoter{Lux}
%	& \longrightarrow
%	\protein{RNApol} + \promoter{Lux} + \protein{Output}
%\]
%}
% https://chem.libretexts.org/Bookshelves/Biological_Chemistry/Supplemental_Modules_(Biological_Chemistry)/Enzymes/Enzymatic_Kinetics/Sigmoid_Kinetics


Kinetics:
\begin{subequations}
\[
	\protein{LuxR^\star} 
	& =
	\frac{
		\signal{3OC6\text{-}HSL}
	}{
		k_{\ref{e:LuxR_Act}} + \signal{3OC6\text{-}HSL}
	}
	\protein{LuxR}_\text{total},
%	\TEXT{initially}
%	\protein{LuxR^\star} = 0
	,
	%
	\\
	%%
	\promoter{Lux^\star} 
	& =
	\frac{
		\protein{LuxR^\star}
	}{
		k_{\ref{e:P_Lux_Act}} + \protein{LuxR^\star}
	}
	\promoter{Lux}_\text{total},
%	\TEXT{initially}
%	\promoter{Lux^\star} = 0
	,
	%
	\\
	%
	\tfrac{\d}{\d{t}}
	\protein{Output}
	& =
	k_{\ref{e:P_Lux_Out}} \promoter{Lux^\star}
	.
\]
\end{subequations}
%
%
In combination, we have
\[
	\tfrac{\d}{\d{t}}
	\protein{Output}
	=
	k_{\ref{e:P_Lux_Out}}
	\promoter{Lux}_\text{total}
	\frac{
		\signal{} \promoter{}_\text{total}
	}{
		k_{\ref{e:LuxR_Act}} k_{\ref{e:P_Lux_Act}}
		+
		k_{\ref{e:P_Lux_Act}} \signal{}
		+
		\signal{} \promoter{}_\text{total}
	}
	.
\]


\subsubsection*{Sal/NahR}

Pathway:
%
\[
	\signal{Sal} \act \protein{NahR} \act \promoter{Sal} \act \protein{Output}
	.
\]

%

According to 
\cite{SchellWender1986, HuangSchell1991},
the transcription activator \protein{NahR}
binds to its recognition site
$-83$ to $-45$,
and does so without the inducer \cite[p.10837]{HuangSchell1991}.
%
The inducer
(here \signal{Sal})
controls the contact near $-35$,
and
enables transcription.
%
%
In \cite{SchellBrownRaju1990},
it was suggested 
that the active configuration of \protein{NahR} is a tetramer,
while \cite{ParkLimShin2005}
reported that 
there could be three different complexes
$\protein{NahR}:\promoter{Sal}$.
%
%
Following \cite{Peking2013},
we suppose
that 
$4 \times \protein{NahR}$ bind to the DNA,
and
transcription starts
once
one $\signal{Sal}$
binds to each \protein{NahR}.
%
For the sensors we assume an abundance
of \protein{NahR}
\TODO{describe constitutive promoter},
hence the inactive form of the promoter
is $\protein{NahR}_4 \with \promoter{Sal}$.
%
%
We write
$
	\promoter{Sal^\star} :=
	\signal{Sal}_4 \with \protein{NahR}_4 \with \promoter{Sal}
$
for the active form.
%
%
Mechanism:
\begin{subequations}
\[
	\label{e:NahR_Act}
	%
	\cee{
		\signal{Sal} + 
		\signal{Sal}_{k - 1} \with \protein{NahR}_4 \with \promoter{Sal}
		& <=>>
		\signal{Sal}_k \with \protein{NahR}_4 \with \promoter{Sal}
	}
	,
	\quad \text{k} = 1, 2, 3, 4,
	%
	\\
	%
	\label{e:NahR_Out}
	%
	\cee{
		\promoter{Sal^\star}
		&
		->
		\promoter{Sal^\star} + \protein{Output}
	}
	.
\]
\end{subequations}
%
%
%
We implement the mechanism as follows:
\[
	\promoter{Sal^\star}
	=
	\frac{
		\signal{}^{n_{\ref{e:NahR_Act}}}
	}{
		k_{\ref{e:NahR_Act}} + \signal{}^{n_{\ref{e:NahR_Act}}}
	}
	\promoter{Sal}_\text{total}
\]



\subsubsection*{pC/rpaR}

Pathway:
%
\[
	\signal{pC} \act \protein{RpaR} \act \promoter{rpa}
	\act 
	\protein{Output}
	.
\]

The \emph{Rhodopseudomonas palustris} transcriptional regulator
\protein{RpaR},
when purified,
binds an inverted repeat element 
centered at $-48.5$bp
of its promoter
\cite{HirakawaETAL2011}.
%
%
Transcription depends on 
the inducer 
\emph{p}-coumaroyl-homoserine lactone,
or \signal{pC} for short.
%
%
%There is
%pC-HSL-RpaR-activated antisense transcription of \protein{rpaR}.
%
%
The complex 
$\signal{pC} : \protein{RpaR}$
when bound to the promoter
activates transcription
\cite[Discussion]{HirakawaETAL2011}.






\footnotesize
\bibliographystyle{unsrt}
\bibliography{refs}

\SHOWTODOS

\leavevmode\vfill{\tiny\color{lightgray}\hfill{Intro Bio Comp (\DTMnow)}}
\end{document}





