\documentclass[12pt]{article}

\usepackage{amsmath, amsfonts}
\usepackage{graphicx}
\usepackage[svgnames]{xcolor}
\usepackage{datetime2}
\usepackage{epstopdf}
\usepackage[colorlinks=true, citecolor={red}, urlcolor={DarkBlue}]{hyperref}
\usepackage{amssymb} 

\usepackage{fullpage}
\setlength{\parindent}{0cm}

% For editing purposes:
%\usepackage[margin=10pt]{geometry}


% https://latex.org/forum/viewtopic.php?t=10456
\usepackage{titlesec}
\titleformat{\subsubsection}[runin]% runin puts it in the same paragraph
{\normalfont\bfseries}% formatting commands to apply to the whole heading
{\thesubsubsection}% the label and number
{}% space between label/number and subsection title
{}% formatting commands applied just to subsection title
[.]% punctuation or other commands following subsection title


\newcommand{\cbra}[1]{{\color{gray}\ensuremath{#1}}}
\newcommand{\signal}[1]{\ensuremath{\cbra{\langle}\mathrm{#1}\cbra{\rangle}}}
\newcommand{\protein}[1]{\ensuremath{\cbra{(}\mathrm{#1}\cbra{)}}}
\newcommand{\promoter}[1]{\ensuremath{\cbra{[}\mathrm{#1}\cbra{]}}}

% https://tex.stackexchange.com/questions/543953/arrow-with-blunted-end-head-in-math-mode
\newcommand{\act}{\ensuremath{\to}}
\newcommand{\rep}{\ensuremath{\mathrel{\raisebox{-.3ex}{\rotatebox{90}{\scalebox{1}[1.2]{$\bot$}}}}}}

\def\[#1\]{\begin{align}\textstyle#1\end{align}}

% https://tex.stackexchange.com/questions/114113/how-to-label-text-with-equation-number
\newcommand{\eqnum}{\leavevmode\hfill\refstepcounter{equation}\textup{{(\theequation)}}}



\begin{document}

\clearpage

\section{Introduction}

\subsection{The game of 15 sticks}

...


\section{Genetic toolbox}


\subsection{Transducers}

--


\subsection{Sensors}

\subsubsection*{3OC6-HSL/LuxR}

Pathway:
\[
	\signal{3OC6-HSL} \act \protein{LuxR} \act \promoter{P_{Lux}} \act \protein{Output}
	.
\]

 
The transcription activator \protein{LuxR}
occurs in Gram-negative bacteria,
incl.~\emph{Vibrio fischeri}.
%
The bacterium is permeable to the (auto)inducer,
here \signal{3OC6-HSL}.
%
The inducer binds to the N-terminal of \protein{LuxR},
which otherwise inhibits its
functional C-terminal \cite{StevensDolanGreenberg1994}.
%
%
The purified C-terminal binds 
upstream of the \emph{lux} box 
(which is centered at $-42.5$bp \cite{EglandGreenberg1999});
however, 
together with the RNA Pol,
it protects the \emph{lux} box and the \emph{lux} operon
promoter
\cite{StevensDolanGreenberg1994}.
%
%
%
Let $\protein{LuxR^\star}$ denote 
the activated form
$\signal{3OC6}:\protein{LuxR}$.
%
%
We assume the mechanism:
%
\begin{subequations}
\[
	\signal{3OC6} + \protein{LuxR}
	& \longleftrightarrow
	\protein{LuxR^\star}
	\\
	\protein{LuxR^\star} + \protein{RNApol} + \promoter{P_{Lux}}
	& \longleftrightarrow
	\protein{LuxR^\star} + \protein{RNApol} : \promoter{P_{Lux}}
	\\
	\protein{RNApol} : \promoter{P_{Lux}}
	& \longrightarrow
	\protein{RNApol} + \promoter{P_{Lux}} + \protein{Output}
	.
\]
\end{subequations}
% https://chem.libretexts.org/Bookshelves/Biological_Chemistry/Supplemental_Modules_(Biological_Chemistry)/Enzymes/Enzymatic_Kinetics/Sigmoid_Kinetics


\subsubsection*{Salicylate/NahR}

Pathway:
%
\[
	\signal{Sal} \act \protein{NahR} \act \promoter{P_{Sal}} \act \protein{Output}
	.
\]

%

According to 
\cite{SchellWender1986, HuangSchell1991},
the transcription activator \protein{NahR}
binds to the recognition site of \promoter{P_{Sal}} at
$-83$ to $-45$,
and does so without the inducer \cite[p.10837]{HuangSchell1991}.
The \protein{NahR} hence activate the expression of \promoter{P_{Sal}}.

%
%
In \cite{SchellBrownRaju1990},
it was suggested 
that the active configuration of \protein{NahR} is a tetramer,
while \cite{ParkLimShin2005}
reported that 
there could be three different complexes
$\protein{NahR}:\promoter{P_{Sal}}$.
Nevertheless, the configuration changes 
then promotes the RNA polymerase binding near $-35$ (upstream of \promoter{P_{Sal}}
%
%
The inducer
(here \signal{Sal}) can induce the conformation change of \protein{NahR},
and activates promoter \promoter{P_{Sal}}.

%
%
Following \cite{Peking2013},
we suppose
that 
$4 \times \protein{NahR}$ bind to the DNA,
and
transcription starts
once
one $\signal{Sal}$
binds to each \protein{NahR}.
%
%
Mechanism (here \protein{NahR^\star} denotes the activated (tetramer) form):
\[
	\signal{Sal} + \protein{NahR}
	& \longleftrightarrow
	\protein{NahR^\star}
	 \\
	 \protein{NahR^\star} + \protein{RNApol} + \promoter{P_{Sal}}
	& \longleftrightarrow
	\protein{NahR^\star} + \protein{RNApol} : \promoter{P_{Sal}}
	\\
	\protein{RNApol} : \promoter{P_{Sal}}
	& \longrightarrow
	\protein{RNApol} + \promoter{P_{Sal}} + \protein{Output}
	
	.
\]


\subsubsection*{pC-HSL/rpaR}

Pathway:
%
\[
	\signal{pC} \act \protein{RpaR} \act \promoter{P_{rpa}}
	\act 
	\protein{Output}
	.
\]

The p-coumaroyl-HSL (\signal{pC-HSL}) /RpaR system works similarly to 
AHL/LuxR system. However, as they come from an anoxygenic 
phototrophic soil bacterium strain, 
the luxIR-type pair, rpaI and rpaR shows good orthogonality to many
QS sensing signals in use \cite{schaefer_et_al_2008}.

The \emph{Rhodopseudomonas palustris} transcriptional regulator
\protein{RpaR},
when purified,
binds an inverted repeat element 
centered at $-48.5$bp
of its promoter
\cite{HirakawaETAL2011}.
%
%
Transcription depends on the inducer 
\emph{p}-coumaroyl-homoserine lactone,
or \signal{pC} for short.
%
%
%There is
%pC-HSL-RpaR-activated antisense transcription of \protein{rpaR}.
%
%
The complex 
$\signal{pC} : \protein{RpaR}$
when bound to the promoter
activates transcription
\cite[Discussion]{HirakawaETAL2011}.




\footnotesize
\bibliographystyle{unsrt}
\bibliography{refs}

\leavevmode\vfill{\tiny\color{lightgray}\hfill{Intro Bio Comp (\DTMnow)}}
\end{document}





